\subsection{Cel projektu i dataset}
Celem projektu jest stworzenie modelu do wykrywania dronów na zdjęciu, który mógłby znaleźć zastosowanie np. na lotniskach, gdzie drony stanowią zagrożenie dla samolotów. Model musi umieć odróżnić drony od samolotów i ptaków, aby nie dawać fałszywych alarmów.

W projekcie użyliśmy zbioru zdjęć \href{https://www.kaggle.com/datasets/cybersimar08/drone-detection}{\textit{Drone Detection}}. Zbiór ten zawiera 4 klasy: 0 - samolot, 1 - dron, 2 - helikopter, 3 - ptak.

\subsection{Data augmentation}

W każdym zadaniu zastosowano augmentację danych. Zastosowano następujące transformacje:
\begin{itemize}
    \item Random Rotation - obrót o losowy kąt z zakresu $(-54^{\circ} , +54^{\circ})$ (factor=0.15)
    \item Random Translation - przesunięcie o losową wartość z zakresu $(-10\%, +10\%)$ (height\_factor=0.1, width\_factor=0.1)
    \item Random Flip - losowe odbicie w poziomie lub pionie
    \item Random Contrast - losowa zmiana kontrastu z zakresu $(-10\%, +10\%)$ (factor=0.1)
\end{itemize}
