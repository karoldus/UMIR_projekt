W projekcie korzystaliśmy z Google Colab. Narzędzie to pozwala w wygodny sposób współpracować online i trenować modele bez posiadania własnej karty graficznej. Niestety darmowa wersja ma spore ograniczenia i czasami brakowało nam zasobów, przez co musieliśmy ograniczyć rozdzielczość zdjęć i czas uczenia. Google Colab korzysta z Notebooków, które pozwalają na wygodne dzielenie kodu na komórki i wykonywanie ich pojedynczo. Dzięki temu można łatwo testować różne fragmenty kodu i sprawdzać wyniki.

W projekcie korzystaliśmy z biblioteki TensorFlow, która pozwala na łatwe tworzenie modeli uczenia maszynowego. W naszym przypadku użyliśmy gotowego modelu EfficientNetB0, który jest dostępny w bibliotece Keras. Model ten jest bardzo wydajny i pozwala na uzyskanie dobrych wyników przy niewielkiej liczbie parametrów.

Do generowania własnych zdjęć wykorzystaliśmy narzędzie Foocus. Jest to zaawansowane narzędzie open source, które pozwala na generowanie zdjęć z różnymi parametrami na własnej karcie graficznej. Narzędzie to jest wygodne w użyciu i pozwala na generowanie dużej ilości zdjęć w krótkim czasie.